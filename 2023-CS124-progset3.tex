\documentclass[11pt]{article}
% \pagestyle{empty}

\usepackage{amsmath,amssymb,amsthm}
\usepackage{booktabs}

\setlength{\oddsidemargin}{-0.25 in}
\setlength{\evensidemargin}{-0.25 in}
\setlength{\topmargin}{-0.9 in}
\setlength{\textwidth}{7.0 in}
\setlength{\textheight}{9.0 in}
\setlength{\headsep}{0.75 in}
\setlength{\parindent}{0.3 in}
\setlength{\parskip}{0.1 in}
\usepackage{epsf}

\def\O{\mathop{\smash{O}}\nolimits}
\def\o{\mathop{\smash{o}}\nolimits}
\newcommand{\e}{{\rm e}}
\newcommand{\R}{{\bf R}}
\newcommand{\Z}{{\bf Z}}

\begin{document}
	
	\section*{CS 124 Programming Assignment 3: Spring 2023}
 		
	\textbf{Your name(s) (up to two): Luke Richey} 
		
	\textbf{Collaborators:} (You shouldn't have any collaborators but the up-to-two of you, but tell us if you did.)

	\textbf{No. of late days used on previous psets: 9}\\
	\textbf{No. of late days used after including this pset: 11}

Homework is due Thursday 2023-04-20 at 11:59pm ET. You are allowed up to {\bf twelve} (college)/{\bf forty} (extension school) late days through the semester, but the number of late days you take on each assignment must be a nonnegative integer at most {\bf two} (college)/{\bf four} (extension school).

For this programming assignment, you will implement a number of
heuristics for solving the {\sc Number Partition} problem, which is
(of course) NP-complete.  As input, the number partition problem
takes a sequence $A = (a_1,a_2,\ldots,a_n)$ of non-negative integers.  The output is a sequence $S = (s_1,s_2,\ldots,s_n)$ 
of signs $s_i \in \{-1,+1\}$ such that the {\em residue}
$$u = \left | \sum_{i=1}^n s_i a_i \right |$$ is minimized.  Another
way to view the problem is the goal is to split the set (or multi-set)
of numbers given by $A$ into two subsets $A_1$ and $A_2$ with roughly
equal sums.  The absolute value of the difference of the sums is the
residue.

As a warm-up exercise, you will first prove that even though Number
Partition is NP-complete, it can be solved in pseudo-polynomial time.
That is, suppose the sequence of terms in $A$ sum up to some number
$b$.  Then each of the numbers in $A$ has at most $\log b$ bits,
so a polynomial time algorithm would take time polynomial in 
$n \log b$.  Instead you should find a dynamic programming algorithm
that takes time polynomial in $nb$.  

\smallskip 
{\bf Give a dynamic programming solution to the Number Partition
problem.}

\textbf{Solution: }

Since the Number Partition problem has overlapping subproblems, it lends itself 
to a dynamic programming solution. Let $D[i][j]$ be a boolean representing 
whether the sum of the first $i$ elements of $A$ is $j$. Then, we will define 
the recurrence of $D[i][j]$ as follows:


\begin{align*}
    D[i][j] = \begin{cases}
        \text{true} & \text{if } i=0,j = 0 \\
        \text{false} & \text{if } i = 0, j \neq 0 \\
        D[i-1][j] & \text{if } j < A[i] \\
        D[i-1][j] \lor D[i-1][j-A[i]] & \text{otherwise} 
    \end{cases}
\end{align*}

Now, we need to actually find these subsets. Let $S[i][j]$ be a subset of 
$A$ consisting of its first $i$ elements that sums to $j$. $D[i][j]$ tells us whether there existence of 
such a subset, but not which one. Then, we will define the recurrence of $S[i][j]$ 
as follows:

\begin{align*}
    S[i][j] = \begin{cases}
        \emptyset & \text{if } i=0 \\
        S[i-1][j] & \text{if } D[i-1][j] = \text{true} \\
        S[i-1][j-A[i]] \cup \{A[i]\} & \text{if } D[i-1][j-A[i]] = \text{true} \\ 
        \emptyset & \text{otherwise} \\
    \end{cases}
\end{align*}

Now that we have defined the recurrences, we can find the partitions. The algorithm 
works as follows:

\begin{enumerate}
    \item For each $i = 0$ to $n$ elements in $A$, we will then iterate through all
    possible sums $j$ from $0$ to $\sum_{k=0}^{k=n} A[k]$, each time calculating
    $D[i][j]$ and $S[i][j]$, where $i$ is the outer loop and $j$ is the inner 
    loop. 

    \item We will then try to find a subset of $A$ of size $n$ that sums to 
    $a = \frac{\sum_{k=0}^{k=n} A[k]}{2}$, by checking if $D[n][a]$
    is true and decrementing $a$ until we find a subset of size $n$ 
    that sums to $a$. Note that we will round $a$ so that it is an integer.

    \item If we find a subset of size $n$ that sums to $a$, then we 
    will return the subset $D[n][a]$ as the partition with the smallest
    residue. Otherwise, we will return the empty set. Additionally, we will return 
    $A \setminus S[n][a]$ as the other partition.    

\end{enumerate}

\textbf{Proof of correctness: }

\begin{proof}
    We will prove the correctness of our algorithm by checking the correctness of 
    our recurrences. 

    \begin{equation*}
        \mathbf{D[i][j]}
    \end{equation*}

    In the base case, that is, $i = 0$ and $j = 0$, we have that $D[0][0] = \text{true}$,
    since the sum of the first $0$ elements of $A$ is $0$. Additionally, we have that
    $D[0][j] = \text{false}$ for all $j \neq 0$, since the sum of the first $0$ elements
    of $A$ is $0$ and $j \neq 0$.

    In the case that $i \neq 0$ and $j < A[i]$, we have that $D[i][j] = D[i-1][j]$,
    since we cannot add $A[i]$ to the sum of the first $i-1$ elements of $A$ to get 
    $j$. We then correctly search for a subset of $A$ of size $i-1$ that sums to $j$.

    In the case that $i \neq 0$ and $j \geq A[i]$, we have that $D[i][j] = D[i-1][j] \lor D[i-1][j-A[i]]$,
    since we can either add $A[i]$ to the sum of the first $i-1$ elements of $A$ to get
    $j$, or we can search for a subset of $A$ of size $i-1$ that sums to $j - A[i]$. We then
    correctly search for a subset of $A$ of size $i-1$ that sums to $j$ or a subset of
    $A$ of size $i-1$ that sums to $j-A[i]$.

    Thus, we have proven that $D[i][j]$ is correctly defined.

    \begin{equation*}
        \mathbf{S[i][j]}
    \end{equation*}

    In the base case, that is, $i = 0$, we have that $S[0][j] = \emptyset$, since
    there is no subset of $A$ of size $0$ that sums to $j$. 

    In the case that $i \neq 0$ and $D[i-1][j] = \text{true}$, we have that $S[i][j] = S[i-1][j]$,
    since there exists a subset of $A$ of size $i-1$ that sums to $j$, by the 
    correctness of $D[i-1][j]$.

    In the case that $i \neq 0$ and $D[i-1][j-A[i]] = \text{true}$, we have that $S[i][j] = S[i-1][j-A[i]] \cup \{A[i]\}$,
    since there exists a subset of $A$ of size $i-1$ that sums to $j-A[i]$ and then
    add $A[i]$ to it to get a subset of $A$ of size $i$ that sums to $j$. 

    Otherwise, we have that $S[i][j] = \emptyset$, since there is no subset of $A$ of size $i$ that sums to $j$.
    Thus, we have proven that $S[i][j]$ is correctly defined.

    We will now prove that our algorithm correctly finds the partition with the 
    smallest residue. We start our search by setting our target sum $a$ to be
    $\frac{\sum_{k=0}^{k=n} A[k]}{2}$. This initial value is correct since the  
    maximum possible sum for the two partitions of $A$ is half of the sum of all the
    elements in $A$. We then check if $D[n][a]$ is true. If it is, then we have
    found a subset of $A$ of size $n$ that sums to $a$. As we decrement $a$, we 
    are bound to find the partition minimizing the residue of $A$.

\end{proof}

We will now prove that, even though the Number Partition problem is NP-complete,
our algorithm runs in pseudopolynomial time.

\begin{proof}
    We will now prove the runtime of our algorithm. Let's first suppose that the 
    elements of $A$ sum to some number $b$. It then follows that each element of 
    $A$ has at most log $b$ bits. 

    In the first step of our algorithm, we must first initialize the arrays $D$ and 
    $S$. There are then $n$ elements summing to $b$, so we initialize both arrays
    to be of size at most $n \times b$. Thus, the initialization of $D$ and $S$ takes
    O($nb$) time.

    Additionally, we must also fill these arrays as described in our algorithm. Since
    there are $nb$ subproblems that we are solving, and each takes O(1) time, then 
    the runtime to fill the arrays is O($nb$).

    Finally, we must also search for the partition with the smallest residue. We 
    calculate $a = \frac{b}{2}$. As stated earlier, each element of $A$ has at most
    log $b$ bits of which there are $n$. Thus, the runtime to compute $b$ is 
    O($n \log(b)$). We then divide $b$ by 2, which takes O($\log b$) time. Rounding 
    $a$ down to the nearest integer takes O(1) time. 

    We then check if $D[n][a]$ is true for the largest possible $a$. In the worst 
    case, we must decrement $a$ until it is 0. Since $a$ is at most $\frac{b}{2}$ 
    the runtime to find $D[n][a]$ is O($b$). We also take the set difference to 
    get the other partition of $A$, which takes O($n$) time. 

    Our overall runtime can be found by summing the runtimes of each step. Thus,
    we have that:

    \begin{equation*}
        \text{Runtime} = O(nb) + O(nb) + O(n \log(b)) + O(\log b) + O(b) + O(n) = O(nb)
    \end{equation*}

\end{proof}


\smallskip 

One deterministic heuristic for the Number Partition problem is the
Karmarkar-Karp algorithm, or the KK algorithm.  This approach uses
{\em differencing}.  The differencing idea is to take two elements
from $A$, call them $a_i$ and $a_j$, and replace the larger by $|a_i -
a_j|$ while replacing the smaller by 0.  The intuition is that if we
decide to put $a_i$ and $a_j$ in different sets, then it is as though
we have one element of size $|a_i - a_j|$ around.  An algorithm based
on differencing repeatedly takes two elements from $A$ and performs a
differencing until there is only one element left; this element equals
an attainable residue.  (A sequence of signs $s_i$ that yields this
residue can be determined from the differencing operations performed
in linear time by two-coloring the graph $(A,E)$ that arises, where
$E$ is the set of pairs $(a_i,a_j)$ that are used in the differencing
steps.  You will not need to construct the $s_i$ for this assignment.)

For the Karmarkar-Karp algorithm suggests repeatedly taking the
largest two elements remaining in $A$ at each step and differencing
them.  For example, if $A$ is intially $(10,8,7,6,5)$, then the KK
algorithm proceeds as follows:
\begin{eqnarray*}
(10,8,7,6,5) & \rightarrow & (2,0,7,6,5) \\
 & \rightarrow & (2,0,1,0,5) \\
 & \rightarrow & (0,0,1,0,3) \\
 & \rightarrow & (0,0,0,0,2) \\
\end{eqnarray*}
Hence the KK algorithm returns a residue of 2.  The best possible
residue for the example is 0.

\smallskip 
{\bf Explain briefly how the Karmarkar-Karp algorithm can be 
implemented in $O(n \log n)$ steps, assuming the values in 
$A$ are small enough that arithmetic operations take one step.}
\smallskip 

\setlength{\fboxsep}{10pt}
\fbox{\parbox{0.9\linewidth}{
    \textbf{Solution: }
    We would like to be able to take the two largest elements in $A$ efficiently. 
    We can do so by using a max-heap, which takes $O(n)$ time to construct. We can 
    then take the two largest elements in $A$ using the heap's extract-max function,
    which will take $O(\log n)$ time. We can then perform the differencing operation
    on these two elements, say $a_i$ and $a_j$ which will take $O(1)$ time. We can 
    then insert these two elements back into the heap, which will take $O(\log n)$ 
    time. We will continue this process while $A$ has at least 2 elements.
    Since there are $n$ elements in $A$ and our operations take 
    $O(\log n)$ time, then the runtime of the KK algorithm is $O(n \log n)$. 
}}
 
\smallskip

You will compare the Karmarkar-Karp algorithm and a variety of
randomized heuristic algorithms on random input sets.  Let us first
discuss two ways to represent solutions to the problem and the state
space based on these representations.  Then we discuss heuristic
search algorithms you will use.

The standard representation of a solution is simply as a sequence $S$
of $+1$ and $-1$ values.  A random solution can be obtained by
generating a random sequence of $n$ such values.  Thinking of all
possible solutions as a state space, a natural way to define neighbors
of a solution $S$ is as the set of all solutions that differ from $S$
in either one or two places.  This has a natural interpretation if we
think of the $+1$ and $-1$ values as determining two subsets $A_1$
and $A_2$ of $A$.  Moving from $S$ to a neighbor is accomplished
either by moving one or two elements from $A_1$ to $A_2$, or moving
one or two elements from $A_2$ to $A_1$, or swapping a pair of
elements where one is in $A_1$ and one is in $A_2$.

A {\em random move} on this state space can be defined as follows.
Choose two random indices $i$ and $j$ from $[1,n]$ with $i
\neq j$.  Set $s_i$ to $-s_i$ and with probability $1/2$, set $s_j$ to
$-s_j$.  

An alternative way to represent a solution called {\em
prepartitioning} is as follows.  We represent a solution by a sequence
$P = \{p_1,p_2,\ldots,p_n\}$ where $p_i \in \{1,\ldots,n\}$.  The
sequence $P$ represents a prepartitioning of the elements of $A$, in
the following way: if $p_i = p_j$, then we enforce the restriction
that $a_i$ and $a_j$ have the same sign.  Equivalently, if $p_i =
p_j$, then $a_i$ and $a_j$ both lie in the same subset, either $A_1$
or $A_2$.

We turn a solution of this form into a solution in the standard
form using two steps:  
\begin{itemize}
\item We derive a new sequence $A'$ from $A$ which enforces the
prepartioning from $P$.  Essentially $A'$ is derived by resetting
$a_i$ to be the sum of all values $j$ with $p_j = i$, using for
example the following pseudocode:

\smallskip
\begin{tabbing}
\quad \quad \quad \= $A' =$ \= $(0,0,\ldots,0)$ \\
\> {\bf for} $j = 1$ to $n$ \\
\> \> $a'_{p_j} = a'_{p_j} + a_j$  \\
\end{tabbing}
\smallskip

\item We run the KK heuristic algorithm on the result $A'$.
\end{itemize}

For example, if $A$ is initially $(10,8,7,6,5)$, the solution $P = (1,2,2,4,5)$
corresponds to the following run of the KK algorithm:
\begin{eqnarray*}
A = (10,8,7,6,5) & \rightarrow & A' = (10,15,0,6,5) \\
(10,15,0,6,5) & \rightarrow & (0,5,0,6,5) \\
 & \rightarrow & (0,0,0,1,5) \\
 & \rightarrow & (0,0,0,0,4) \\
\end{eqnarray*}
Hence in this case the solution $P$ has a residue of 4.

Notice that all possible solution sequences $S$ can be generated using this prepartition representation, as any split of $A$ into
sets $A_1$ and $A_2$ can be obtained by initially assigning 
$p_i$ to 1 for all $a_i \in A_1$ and similarly assigning 
$p_i$ to 2 for all $a_i \in A_2$.

A random solution can be obtained by generating a sequence of $n$
values in the range $[1,n]$ and using this for $P$.  Thinking of all
possible solutions as a state space, a natural way to define neighbors
of a solution $P$ is as the set of all solutions that differ from $P$
in just one place.  The interpretation is that we change the
prepartitioning by changing the partition of one element.
A {\em random move} on this state space can be defined as follows.
Choose two random indices $i$ and $j$ from $[1,n]$ with $p_i
\neq j$ and set $p_i$ to $j$.


You will try each of the following three algorithms for both
representations.

\begin{itemize}
\item Repeated random:  repeatedly generate random solutions
to the problem, as determined by the representation.
\smallskip
\begin{tabbing}
\quad \quad \quad \= Start \= with a random solution $S$ \\
\> {\bf for} iter = 1 to max\_iter \\
\> \> $S' =$ a random solution \\
\> \> {\bf if} residue$(S') <$ residue$(S)$ {\bf then} $S = S'$ \\
\> return $S$
\end{tabbing}
\smallskip
\item Hill climbing:  generate a random solution to the problem, and
then attempt to improve it through moves to better neighbors.
\smallskip
\begin{tabbing}
\quad \quad \quad \= Start \= with a random solution $S$ \\
\> {\bf for} iter = 1 to max\_iter \\
\> \> $S' =$ a random neighbor of $S$ \\
\> \> {\bf if} residue$(S') <$ residue$(S)$ {\bf then} $S = S'$ \\
\> return $S$
\end{tabbing}
\smallskip
\item Simulated annealing:  generate a random solution to the problem, and
then attempt to improve it through moves to neighbors, that are not always 
better.
\begin{tabbing}
\quad \quad \quad \= Start \= with a random solution $S$ \\
\> $S'' = S$ \\
\> {\bf for} iter = 1 to max\_iter \\
\> \> $S' =$ a random neighbor of $S$ \\
\> \> {\bf if} residue$(S') <$ residue$(S)$ {\bf then} $S = S'$ \\
\> \> {\bf else} $S = S'$ with probability exp($-$(res($S'$)-res($S$))/T(iter))\\
\> \> {\bf if} residue$(S) <$ residue$(S'')$ {\bf then} $S'' = S$ \\
\> return $S''$
\end{tabbing}
\smallskip
Note that for simulated annealing we have the code return the best
solution seen thus far.
\end{itemize}

You will run experiments on sets of 100 integers, with each integer
being a random number chosen uniformly from the range $[1,10^{12}]$.
Note that these are big numbers.  You should use 64 bit integers.  Pay
attention to things like whether your random number generator works on
ranges this large!

Below is the main problem of the assignment.

\smallskip 
{\bf First, write a routine that takes three arguments: a flag, an algorithm code (see Table~1), and an input file. We'll run typical commands to compile and execute your code, as in programming assignment 2; for example, for C/C++, the run command will look as follows:

\noindent \$ ./partition flag algorithm inputfile 

The flag is meant to provide you some flexibility; the autograder will only pass 0 as the flag but you may use other values for your own testing, debugging, or extensions. The algorithm argument is one of the values specified in Table 1. You can also assume the inputfile is a list of 100 (unsorted) integers, one per line. The desired output is the residue obtained by running the specified algorithm with these 100 numbers as input.}

\begin{table}[h]
    \centering
    \begin{tabular}{|c|c|}
        \hline
        Code & Algorithm \\
        \hline
        0 & Karmarkar-Karp\\
        1 & Repeated Random\\
        2 & Hill Climbing\\
        3 & Simulated Annealing\\
        11 & Prepartitioned Repeated Random\\
        12 & Prepartitioned Hill Climbing\\
        13 & Prepartitioned Simulated Annealing\\
        \hline
    \end{tabular}
    \caption{Algorithm command-line argument values}
    \label{tab:algorithm}
\end{table}

\smallskip 

If you wish to use a programming language other than Python, C++, C, Java, and Go, please contact us first.
As before, you should submit either 1) a single source file named one of partition.py, partition.c, partition.cpp, partition.java, Partition.java, or partition.go, or 2) possibly multiple
source files named whatever you like, along with a Makefile (named makefile or Makefile).

\smallskip 
{\bf Second, generate 50 random instances of the problem as described above.
For each instance, find the result from using the Karmarkar-Karp
algorithm.  Also, for each instance, run a repeated random, a hill
climbing, and a simulated annealing algorithm, using both
representations, each for at least 25,000 iterations.  Give tables and/or
graphs
clearly demonstrating the results.  Compare the
results and discuss.  }

\setlength{\fboxsep}{10pt}
\fbox{\parbox{0.9\linewidth}{
    I performed my experiments as described above. I then calculated the average 
    residue from each of the algorithms for each of the 50 instances as well as 
    ran each randomized algorithm over 25,000 iterations. Here are the averages 
    I found:

    \begin{center}
        \begin{tabular}{ll}
        \hline
        Algorithm & Avg. Residue \\
        \hline
        Karmarkar-Karp & 366,685.78 \\
        Repeated Random & 276,406,914.02 \\
        Hill Climbing & 719,607,853.58 \\
        Simulated Annealing & 575,873,017.02 \\
        Prepartitioned Repeated Random & 140.58 \\
        Prepartitioned Hill Climbing & 182.46 \\
        Prepartitioned Simulated Annealing &  223.1 \\
        \hline
        \end{tabular}
    \end{center}

    \smallskip 

    We can clearly see that the prepartioned versions of the algorithms performed 
    much better than the non-prepartitioned versions. A possible explanation for this
    is that we search over a smaller space when we prepartition the numbers. Normally,
    we would have to search over subsets. Because of the smaller search, we are able
    to find more optimal solutions. 

    We can see that for the standard versions, Hill Climbing performed the worst 
    by far with Repeated Random performing the best. This is interesting, because 
    intuitively, I would have expected Simulated Annealing to perform the best since
    it moves closer to a better solution and has a mechanism to escape local minima.
    So it is really interesting that the Repeated Random algorithm performed the best 
    in this case, which seems to me is just a testament to the difficulty of the 
    problem. 

    I found similar results among the prepartitioned versions. Once again, repeated
    random performed the best, but this time, it was much closer to the other algorithms.
    I think one possible explanation for this is that the prepartitioned versions
    are much easier to solve. So the algorithms are able to converge more quickly 
    and thus end up converging to similar values. We do see that Simulated Annealing
    performed the worst in this case, perhaps because the random chance of moving 
    away from an optimal solution ended up harming its performance.

    Finally, we see that the Karmarkar-Karp algorithm performed in the middle of 
    the group. This makes sense because it is a greedy algorithm that is not
    randomized and so it just finds an approximate solution.
}}

For the simulated annealing algorithm, you must choose a {\em cooling
schedule.}  That is, you must choose a function T(iter).  We suggest
T(iter) = $10^{10}(0.8)^{\lfloor \mbox{iter} / 300 \rfloor}$ for
numbers in the range $[1,10^{12}]$, but you can experiment with this
as you please.  

Note that, in our random experiments, we began with a random initial
starting point.

{\bf Discuss briefly how you could use the solution from the
Karmarkar-Karp algorithm as a starting point for the randomized
algorithms, and suggest what effect that might have. 
(No experiments are necessary.)}

\setlength{\fboxsep}{10pt}
\fbox{\parbox{0.9\linewidth}{
    \textbf{Solution: }

        Currently, we generate a random solution for each algorithm. We could have a 
    much better initial approximation if we ran the Karmarkar-Karp algorithm and
    used the result as the initial solution for the randomized algorithms. That way,
    the randomized algorithms would start closer to the solution state and would 
    be able to find better solutions, faster. 

    \smallskip 

        However, this would not affect the repeated random algorithm, since it generates a 
    random solution each time. As such, the initial solution does not really matter.
    However, it would affect the hill climbing and simulated annealing algorithms
    since they only accept new solutions if they are better than the current solution.  
    As such, they would perform better if they ran Karmarkar-Karp first. However,
    this would increase the runtime of the algorithms, as instead of just generating 
    random numbers we would have to run all of the Karmarkar-Karp algorithm first.

}}


Finally, the following is entirely optional;  you'll get no credit for it.
But if you want to try something else, it's interesting to do.

{\bf Optional:} Can you design a BubbleSearch-based heuristic for this
problem?
The Karmarkar-Karp algorithm greedily takes the top two items at each
step, takes their difference, and adds that difference back into the
list of numbers.  A BubbleSearch variant would not necessarily take
the top two items in the list, but probabilistically take two items
close to the top.  (For instance, you might ``flip coins'' until the
the first heads;  the number of flips (modulo the number of items) 
gives your first item.  Then do the same, starting from where you
left off, to obtain the second item.  Once you're down to a small
number of numbers -- five to ten -- you might want to switch back
to the standard Karmarkar-Karp algorithm.)  Unlike the original 
Karmarkar-Karp algorithm, you can repeat this algorithm multiple
times and get different answers, much like the Repeated Random algorithm
you tried for the assignment.  Test your BubbleSearch algorithm
against the other algorithms you have tried.  How does it compare?

\end{document}




